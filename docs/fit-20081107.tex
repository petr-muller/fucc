\documentclass{beamer}

\usepackage{beamerthemesplit}
\usepackage[czech]{babel}
\usepackage[utf8]{inputenc}
\usetheme{Frankfurt}

\title{Fuzz testování překladačů}
\author{Petr Muller}
\date{\today}

\begin{document}

\frame{\titlepage}

\section[Obsah]{}
\frame{\tableofcontents}
\section{Úvod}
\frame
{
  \frametitle{Pozadí práce}
  \begin{block}{Práce}
  \begin{itemize}
    \item Bakalářská práce
    \item Cílem bylo zjistit, zda je možné fuzz testování aplikovat na překladače
    \item Byl implementován nástroj využívající tuto metodu
  \end{itemize}
  \end{block}
}

\section{Fuzz testování}
\frame
{
  \frametitle{Metoda fuzz testování}
  \begin{itemize}
    \item Jednoduchá automatizovaná metoda pro testování robustnosti aplikací
    \item Princip: \begin{itemize}
      \item Jako vstup programu se použije (pseudo) náhodný řetezec znaků
      \item Sleduje se, zda program zvládne takový stav zpracovat
      \item Pokud program zareaguje, testem prošel. Pokud ne (pád, zacyklení), pak neprošel
      \end{itemize}
    \item Metoda zvládá rychle odhalit nedostatečné ošetření vstupu
    \item Užitečné při testování, jak program nakládá s nedůvěryhodnými (vzdálenými) vstupy
    \item Nedostatečné ošetření vstupu je hlavním zdrojem bezpečnostních problémů software
  \end{itemize}
}

\frame
{
  \frametitle{Typy fuzz testování}
  \begin{block}{Čistý fuzz}
  \begin{itemize}
    \item Použije se zcela náhodný řetězec
    \item Triviální implementace
    \item Testuje pouze \"povrch\" programu
  \end{itemize}
  \end{block}
  \begin{block}{Fuzzing formátu}
  \begin{itemize}
    \item Se používá v případě, že vstup je v nějakém jazyce nebo protokolu (HTML, prog. jazyk...)
    \item Vstupem je náhodná, avšak gramaticky správná věta jazyka vstupu
    \item Vstup projde "hlouběji" do testovaného programu
  \end{itemize}
  \end{block}
}

\frame
{
  \frametitle{Fuzz testování překladačů}
  \begin{block}{Čistý fuzz}
  \begin{itemize}
  \item Nevhodné, testuje pouze lexikální analyzátor
  \end{itemize}
  \end{block}
  \begin{block}{Fuzz jazyka C}
  \begin{itemize}
    \item Náhodné, avšak validní programy v jazyce C
    \item Testují odolnost velké části kódu překladače
    \item Je možné touto metodou nalézt chyby, které se projevují v čase překladu
    \end{itemize}
  \end{block}
}

\frame
{
  \frametitle{Chyby v překladačích}
  \begin{block}{Méně závažné}
  Nebrání správnému překladu (chybějící/přebývající varování, výkonnostní problémy)
  \end{block}
  \begin{block}{Středně závažné}
  \begin{itemize}
  \item Projevují se při překladu
  \item Brání překladu (odmítnutí platného kódu, ICE\footnote{Internal Compiler Error})
  \end{itemize}
  \end{block}
  \begin{block}{Velmi závažné chyby}
  \begin{itemize}
  \item Projevují se až ve špatném chování výsledného programu
  \item Obtížné odhalit pravou příčinu (přijetí neplatného kódu, generování špatného kódu)
  \end{itemize}
  \end{block}
}

\frame
{
  \frametitle{Hledání nejzávažnějších chyb pomocí fuzz testování}
  \begin{block}{ }
  \begin{itemize}
  \item Není možné určit konkrétní správný výstup překladače
  \item Správnost výstupu překladače je dána správností výstupu výsledného programu
  \item Jak určit správnost výstupu náhodného programu?
  \end{itemize}
  \end{block}
  \begin{block}{Fuzz testování překladače s porovnáváním}
  
  Náhodný program musí mít deterministický výstup
  \scriptsize{
  \begin{enumerate}
    \item Program je přeložen dvakrát: testovaným a referenčním překladačem
    \item Oba přeložené programy se spustí a porovná se jejich výstup
    \item Je nepravděpodobné, že se stejná chyba vyskytuje v obou překladačích
    \item Pokud se výstupy liší, pak se v jednom z překladačů nachází chyba
    \end{enumerate}
  }
  \end{block}
}

\section{Stavba generátoru vět jazyka C}
\frame
{
}


\section{Nástroj fucc}
\frame
{
\frametitle{fucc - Fuzzing C Compiler}
  \begin{block}{Nástroj pro realizaci popsané metody}
  \begin{itemize}
    \item Generátor náhodných vět 
    \item Překlad oběma překladači
    \item Tvorba potřebných výstupů
    \item Porovnávání výstupů
  \end{itemize}
  \end{block}

  \begin{block}{Projekt}
  \begin{itemize}
    \item Implementováno v Pythonu + shell scripty
    \item Open-source, repozitář http://git.afri.cz/git/fucc.git
  \end{itemize}
  \end{block} 
}

\frame
{
  \frametitle{Aktuální vlastnosti}
  \begin{block}{Vlastnosti}
    \begin{itemize}
      \item Generátor je schopen tvořit platné, ne zbytečně složité programy
      \item Některé části lze konfigurovat
      \item Hledá rozdíly ve výsledku překladu, výstupu a ukončení programu
      \item Je schopen rozeznat některé obvyklé chyby generátoru
    \end{itemize}
  \end{block}

  \begin{block}{Výsledky}
  \begin{itemize}
    \item Program byl testován pouze jednou, nalezeno bylo cca 10 chyb v TCC, a dvou verzích GCC
  \end{itemize}
  \end{block}
}

\frame
{
  \frametitle{Budoucí vývoj}
  \begin{block}{Odstranění chyb}
    \begin{itemize}
    \item Odstranit některé případy vygenerovaného neplatného kódu
    \item Lépe integrovat jednotlivé součásti
    \item Zredukovat jazykově závislou část generátoru
    \end{itemize}
  \end{block}
  \begin{block}{Vlastnosti}
  \begin{itemize}
  \item Lepší konfigurovatelnost porovnávání
  \item Usnadnění ověřování chyb
  \item Podpora regresního testování
  \item Grafické uživatelské rozhraní
  \end{itemize}
  \end{block}
}

\section{Závěr}
\frame
{
  \frametitle{Závěr}
  \begin{block}{Závěr}
  \center{
  Otázky?

  Děkuji za pozornost!
  }
  \end{block}
}

\end{document}
